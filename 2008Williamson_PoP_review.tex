%%%%%%%%%%%%%%%%%%%%%%%%%%%%%%%%%%%%%%%%%
% Article Notes
% LaTeX Template
% Version 1.0 (1/10/15)
%
% This template has been downloaded from:
% http://www.LaTeXTemplates.com
%
% Authors:
% Vel (vel@latextemplates.com)
% Christopher Eliot (christopher.eliot@hofstra.edu)
% Anthony Dardis (anthony.dardis@hofstra.edu)
%
% License:
% CC BY-NC-SA 3.0 (http://creativecommons.org/licenses/by-nc-sa/3.0/)
%
%%%%%%%%%%%%%%%%%%%%%%%%%%%%%%%%%%%%%%%%%

%----------------------------------------------------------------------------------------
%	PACKAGES AND OTHER DOCUMENT CONFIGURATIONS
%----------------------------------------------------------------------------------------

\documentclass[
10pt, % Default font size is 10pt, can alternatively be 11pt or 12pt
a4paper, % Alternatively letterpaper for US letter
twocolumn, % Alternatively onecolumn
landscape % Alternatively portrait
]{article}

\input{structure.tex} % Input the file specifying the document layout and structure
\usepackage{hyperref}


%----------------------------------------------------------------------------------------
%	ARTICLE INFORMATION
%----------------------------------------------------------------------------------------

\articletitle{Previews:  \textit{The Philosophy of Philosophy}} % The title of the article
\articlecitation{\cite{Williamson2007}} % The BibTeX citation key from your bibliography

\datenotesstarted{August 27, 2018} % The date when these notes were first made
\docdate{\datenotesstarted; rev. \today} % The date when the notes were lasted updated (automatically the current date)
\docauthor{Summarized and commented by Shimpei Endo} % Your name

%----------------------------------------------------------------------------------------

\begin{document}

\pagestyle{myheadings} % Use custom headers
\markright{\doctitle} % Place the article information into the header

%----------------------------------------------------------------------------------------
%	PRINT ARTICLE INFORMATION
%----------------------------------------------------------------------------------------

\thispagestyle{plain} % Plain formatting on the first page

\printtitle % Print the title

%----------------------------------------------------------------------------------------
%	ARTICLE NOTES
%----------------------------------------------------------------------------------------

\section*{In a nutshell... }
\cite{Williamson2007}
\begin{itemize}
	\item	rejects the ideology of ``linguistic (and conceptual) turn'', which is a distinctive regime which seems to characterize the contemporary Anglo speaking philosophy,
	\item	claims that what we do when we do philosophy is to argue counterfactuals, and
	\item	suggests to see philosophy as an armchair science.  
\end{itemize}

\paragraph{Keywords:} 
methodology, science, counterfactuals, linguistic turn

%------------------------------------------------


\section*{Bibliographical notes} % Unnumbered section

\paragraph{About the author.}
This monograph \cite{Williamson2007} was written by Timothy Williamson and published in 2007. 
A summary by the author himself \cite{williamson2010} is available. 
Timothy Williamson is Wykeham Professor of Logic at the University of Oxford, and Fellow at New College, Oxford.
In a recent interview (  \url{http://www.whatisitliketobeaphilosopher.com/timothy-williamson/} ), he talked about his background, which may have something to do with what he as a philosopher does.
His forthcoming book \textit{Doing Philosophy} deals with the similar topics. 

\paragraph{Reviews.}
\cite{Russell2010} wrote a relatively long and affirmative review, which covers most topics mentioned in this book. 
\cite{Kornblith2009} casts questions on philosophical methodologies which Willaimson did not explicitly address in his book. 
\cite{Eklund2010} is a messy and less-organized but short review with several disagreements. 
\cite{Hacker2008} writes a critique so harsh on Williamson that he switches the title: from ``\emph{the} philosophy of philosophy'' into  ``\emph{a} philosophy of philosophy'', implying that the scope of this book covers at most a particular philosopher Timothy Williamson (i.e. \emph{his} philosophy). 

%------------------------------------------------

\section*{Against ``linguistic turn''}
Williamson criticizes the observation that our philosophical practices are after the ``linguistic (or conceptual) turn''.
Williamson challenges this commonly held belief that our contemporary Anglo-speaking so called analytic philosophy is characterized as linguistic or conceptual philosophy -- philosophical issues are ultimately issues of our language or concepts and its use. Chapter 1 through 4 of this book argue that it is wrong. 
To summarize linguistic or conceptual era, there are at least three common characterizations among them. 
(i) arm chair, (ii) analyticity, necessity, a prioricity, and (iii) domains exclusive and particular to philosophy. 
Summing up, such a characterization was due to their self understanding that philosophy is very unlike to sciences. 
Among them, Williamson devotes several chapters to critically analyze analicity. 


\section*{Towards counterfactuals}
When we do philosophy, according to the previous discussion of Williamson, 
we are \emph{not} doing anything with language or concept necessarily nor generally. 
Then what are we doing? 
Williamson's answer is: \emph{counterfactuals}. 
Counterfactuals play an imporntant role when we do \emph{thought experiments}: philosophers often (or some always) imagine counterfactual but possible scenario or even counterpossible (i.e. impossible) situations. 
Williamson demonstrates that our ordinary cognitive abilities and thought experiments work in philosophical investigations with a model case of the Gettier case. 
Williamson even claims that modal metaphysics is just a special case of counterfactuals (cf.\cite{Williamson2013a}). 

Williamosn further discusses epistemology of counterfactuals: how do we know things against fact? 
This move might synchoronize his move from metaphysical analycity to epistemic one. 

\section*{For good philosophy}
\paragraph{We can do better.}
\begin{quotation}
\noindent A reasonable hypothesis is that our current methodology is good enough to generate progress in philosophy, but not by much: 10 steps forward, 9 steps back. Nevertheless we can improve our performance even without radically new methods. [...]
(p. 7-8)
\end{quotation}

\paragraph{Be patient (as scientists be)}
\begin{quotation}
\noindent [...] Imparience with the long haul of technical reflection is a form of shallowness, often thinly disguised by histrionic advocacy of depth. 
Serious philosophy is always likely to bore those with short attention spans. (p. 44-45)
\end{quotation}

\section*{Evaluation}
\paragraph{Stick with the first order.}
Williamson outlines a philosophical methodology which deals with the question of what philosophy is without hiking up to ``meta'' level. 
Average thinkers always abandon the object level and run away to the higher order, even though they do not know what is going to happen there. 

\paragraph{Powerfully alive.}
Williamson is not only influential but also still alive and active. He even annouced his next book \textit{Doing Philosophy} coming out soon. 
It woule be benefitical to read this grouding work in advance to the publication of the new one. 
%----------------------------------------------------------------------------------------
%	BIBLIOGRAPHY
%----------------------------------------------------------------------------------------

\renewcommand{\refname}{Reference} % Change the default bibliography title

\bibliography{Mendeley} % Input your bibliography file

%----------------------------------------------------------------------------------------

\end{document}