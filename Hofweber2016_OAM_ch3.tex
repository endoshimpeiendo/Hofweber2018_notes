%%%%%%%%%%%%%%%%%%%%%%%%%%%%%%%%%%%%%%%%%
% Article Notes
% LaTeX Template
% Version 1.0 (1/10/15)
%
% This template has been downloaded from:
% http://www.LaTeXTemplates.com
%
% Authors:
% Vel (vel@latextemplates.com)
% Christopher Eliot (christopher.eliot@hofstra.edu)
% Anthony Dardis (anthony.dardis@hofstra.edu)
%
% License:
% CC BY-NC-SA 3.0 (http://creativecommons.org/licenses/by-nc-sa/3.0/)
%
%%%%%%%%%%%%%%%%%%%%%%%%%%%%%%%%%%%%%%%%%

%----------------------------------------------------------------------------------------
%	PACKAGES AND OTHER DOCUMENT CONFIGURATIONS
%----------------------------------------------------------------------------------------

\documentclass[
10pt, % Default font size is 10pt, can alternatively be 11pt or 12pt
a4paper, % Alternatively letterpaper for US letter
twocolumn, % Alternatively onecolumn
landscape % Alternatively portrait
]{article}

%%%%%%%%%%%%%%%%%%%%%%%%%%%%%%%%%%%%%%%%%
% Article Notes
% Structure Specification File
% Version 1.0 (1/10/15)
%
% This file has been downloaded from:
% http://www.LaTeXTemplates.com
%
% Authors:
% Vel (vel@latextemplates.com)
% Christopher Eliot (christopher.eliot@hofstra.edu)
% Anthony Dardis (anthony.dardis@hofstra.edu)
%
% License:
% CC BY-NC-SA 3.0 (http://creativecommons.org/licenses/by-nc-sa/3.0/)
%
%%%%%%%%%%%%%%%%%%%%%%%%%%%%%%%%%%%%%%%%%

%----------------------------------------------------------------------------------------
%	REQUIRED PACKAGES
%----------------------------------------------------------------------------------------

\usepackage[includeheadfoot,columnsep=2cm, left=1in, right=1in, top=.5in, bottom=.5in]{geometry} % Margins

\usepackage[T1]{fontenc} % For international characters
\usepackage{XCharter} % XCharter as the main font

\usepackage{natbib} % Use natbib to manage the reference
\bibliographystyle{apalike} % Citation style

\usepackage[english]{babel} % Use english by default

%----------------------------------------------------------------------------------------
%	CUSTOM COMMANDS
%----------------------------------------------------------------------------------------

\newcommand{\articletitle}[1]{\renewcommand{\articletitle}{#1}} % Define a command for storing the article title
\newcommand{\articlecitation}[1]{\renewcommand{\articlecitation}{#1}} % Define a command for storing the article citation
\newcommand{\doctitle}{\articlecitation\ --- ``\articletitle''} % Define a command to store the article information as it will appear in the title and header

\newcommand{\datenotesstarted}[1]{\renewcommand{\datenotesstarted}{#1}} % Define a command to store the date when notes were first made
\newcommand{\docdate}[1]{\renewcommand{\docdate}{#1}} % Define a command to store the date line in the title

\newcommand{\docauthor}[1]{\renewcommand{\docauthor}{#1}} % Define a command for storing the article notes author

% Define a command for the structure of the document title
\newcommand{\printtitle}{
\begin{center}
\textbf{\Large{\doctitle}}

\docdate

\docauthor
\end{center}
}

%----------------------------------------------------------------------------------------
%	STRUCTURE MODIFICATIONS
%----------------------------------------------------------------------------------------

\setlength{\parskip}{3pt} % Slightly increase spacing between paragraphs

% Uncomment to center section titles
%\usepackage{sectsty}
%\sectionfont{\centering}

% Uncomment for Roman numerals for section numbers
%\renewcommand\thesection{\Roman{section}}
 % Input the file specifying the document layout and structure
\usepackage{hyperref}


%----------------------------------------------------------------------------------------
%	ARTICLE INFORMATION
%----------------------------------------------------------------------------------------

\articletitle{Ch.3 Quantification} % The title of the article
\articlecitation{\cite{Hofweber2016_OntologyAmbitions}} % The BibTeX citation key from your bibliography

\datenotesstarted{February 14, 2019} % The date when these notes were first made
\docdate{\datenotesstarted; rev. \today} % The date when the notes were lasted updated (automatically the current date)
\docauthor{Summarized and commented by Shimpei Endo} % Your name

%----------------------------------------------------------------------------------------

\begin{document}

\pagestyle{myheadings} % Use custom headers
\markright{\doctitle} % Place the article information into the header

%----------------------------------------------------------------------------------------
%	PRINT ARTICLE INFORMATION
%----------------------------------------------------------------------------------------

\thispagestyle{plain} % Plain formatting on the first page

\printtitle % Print the title

%----------------------------------------------------------------------------------------
%	ARTICLE NOTES
%----------------------------------------------------------------------------------------

\section*{In a nutshell... }
The third chapter of \cite{Hofweber2016_OntologyAmbitions}
	defends that quantification plays an important role in ontology, and
	shows at least two readings of quantifiers: \emph{external} and \emph{internal}.

\noindent \textbf{Keywords:}
quantification, quantifiers, internal, external, polysemy

%------------------------------------------------

\section*{Comments}
\paragraph{A TA used!}
Hofweber implicitly employs a transcendental argument.

\section*{3-1. The significance of Quantification.
}
\paragraph{What is the goal of this chapter?}
The main task of this chapter is to defend a view that
quantification is important to ontology.
There are two quick reasons for this significance:
(1) Ontological questions are asked in a form of ``Are there Fs'', and
(2) Quantified sentences answer these questions.
However, both points are criticized for being trivial i.e. not informative.
For instance, Aristotelian metaphysicians would reject (1).
Hofweber rescues the importance of quantification from these criticisms
but it turns out to be a more complicated way than it was thought.

\section*{3.2 The (Fairly) Uncontroversial Facts about Quantifiers}
What can we say about quantifiers safely?
This section builds a common ground of how quantifiers work.

\paragraph{3.2.1. The Structure of Quantified Noun Phrases}
Examples of quantifiers include: \emph{some} and \emph{every}.
Note the difference from quantifiers in logic (viz., first order predicate logic). Quantifiers in natural languages are so rich that linguists have invented a more general system called GQT.

\paragraph{3.2.2. ``There Is'' Is Not a Quantifier}
Some, including Lewis 2004, misunderstand that an expression ``there is'' is
a quantifier.
Wrong. ``There are numbers.'' contains quantifier.
But it is not ``there are'' but ``s''.

\paragraph{3.2.3. Singular and plural quantifiers}
Hofweber discusses the difference between \emph{all} and \emph{every}.

\paragraph{3.2.4. The Domain Condition Reading}
Later called as extensional reading.
One of the roles of quantifers is to make our languages to express the most general
features of the world.
Of course, many disputes have been occured. Hofweber tries to draw a safe starting point.

\section*{3.3. Semantic Underspecification}
Semantic underspecification often happens in natural languages.
This section overviews this phenomenon and set aside unimportant issues.

\paragraph{Preparation: to be precise.}
 Hofweber takes a \emph{sentence} as ``a string of words with basic argument [syntactic] structures'' [p. 60]. For example, ``He gaves her cat food'' can be two different sentences. Hofweber also introduces a distinction between the content of the sentence uttered and the content of the utterance of that sentence [p.61].

 \paragraph{Observation.}
 Hofweber then moves to an observation of common cases of semantic underspecification.

 \begin{enumerate}
 	\item [Gentire] ``John\emph{'s} car is fast''. John's possesion? Being driven by John? John sitting in its backseat?
	\item[Plural] ``Four philosophers carried two pianos''. Each carries two? Or four carry two together?
	\item[Reciprocals]
	``The exists on the Santa Monica Freeway are less than a mile from each other''.
	\item[Polysemy]
	``Before I \emph{get} home I have to \emph{get} some beer to \emph{get} drunk.''
	Polysemy is the most important, discussed in the next section. Notice the difference from \emph{ambiguity}. For instance, two very different things are assigned to a word \emph{bank} by coincidence, which does not make any semantic underspecification.
 \end{enumerate}

 \section*{3.4 Quantifers as a Source of Underspecification}
 This section claims that quantifers are polysemous, which leads underspecification.
 Hofweber's tasks are i) to observe how truth values vary due to polysemous nature of quantifiers and ii) to show some evidence of quantifers being polysemous.

 \paragraph{3.4.1 How to Settle the Issue}
 Hofweber's motivatioon is quite empirical.
 He starts with observation of natural language actually spoken. Only after that, Hofweber calls philosophy.

 Hofweber rejects the method of cases for its causing endless disputes. Instead, Hofweber concentrates on the use and role in daily communications. By observing how we usually use quantifiers, Hofweber tries to conclude there are at least two different functions (or roles) are required.

 \paragraph{3.4.2 The Communication Function of Quantifers}
 Hofweber adds internal (or inferential) function of quantifiers on top of external reading (given in 3.2.4).
 Towards importing such a new category, Hofweber offered three steps or argument:
 \begin{enumerate}
 	\item In daily life, we need not only the syntaxtic but also an inferential function.
	\item	What satisfy (1) are quantifers.
	\item	External (domain condition) reading does not suffice (1).
 \end{enumerate}

 Hofweber demonstrates some instances of the first and second argument: communicating under partial ignorance. Assume that Fred  loves Reagan, who loved by many Republicans. Also suppose that you forgets who satisfies the position of Reagan. But still, asking ``Fred admires somebody. But who was that?'' implies (or commits) there is someone (exisitence!) Fred admires.
 More generally, F(t) implies ``Something is F''.

 \paragraph{3.4.3. ??}
 Hofweber discusses the third task: blocks external readings do the same job.
 His key argument relies on \emph{empty names} such as Sharlock Holmes, setting aside Meinongians.
 To defend the idea of empty names, Hofweber focuses on how ordinary communication goes, particularly how names function [p. 69-]. Hofweber highlights failure of names.

\paragraph{3.4.4 Some examples}
\footnote{Stop! What did you just say about methodology? Why are you talking in terms of examples?}
``Everything exists.'' has two different readings (i.e. truth values) due to two different readings of quantifiers. If you follow the inferential role reading, this is false; it suffices to cast a counterexample (say, Santa) for rejecting it.
On the other hand, according to domain condition reading, it is true. Note that  from everything is F, you cannot lead F(t) for any t.
A less philosophically loaded example is ``I need an assistant''.

\paragraph{3.4.5 Inferential Role and Truth Conditions}
Hofweber first takes some distance from disputes over semantics. It does not matter whether one takes classical (i.e. referential-world/word) or alternative semantic theories.

What is the contribution of inferential role then?
Consider (SomeF) Something is F. The inferential role of quantifer (some).
F(t) is, simply, read in the form of  disjunctions $F(t_1) \lor F(t_2) \lor F(t_3)...$.
Inferential role determines a) what are instances of this quantifiered sentence and b) how many are they.
\footnote{I do not see why cardinality matters here. }
Formally, easy: inductively defined. But in natural language, things do not go that easy.
\footnote{Quineans criticized!}
Gramattical categories (e.g. names just argued) or location or gramattical structures fail or cause troublesome arguments on what terms are.

\section*{3.5 Compare and Contrast}
This section compares what presented with previous (similar) attempts.

\paragraph{3.5.1 Substituitional Quantifiers}
Objectual vs. substitutional readings are often understood as competiting.
However, Hofweber sets aside this dispute because it is nonsense.
Hofwebers has four reasons. First, such a distinction does not matter to ontology.
Second, substituitonal reading is intended to avoid heavy ontological commitment but, Hofweber believes, such an attempt fails. Substitutonal reading is a ``nominalists' free lunch''. Third. Forth.

\paragraph{3.5.2 Meinong and Non-existent}
Meinongians seem to successfully explain semantic undeterdetermination (such as 110) by settling the distinction between unrestectedly and restectedly selecting domains. In other words, inferential role would be implemented in the restricted kind.

Accepting its success , Hofweber refuses Meinongianism because it should not be the best explanation for needs in our daily communication.
More particularly, first,
Second, ...

\paragraph{3.5.3 Carnap on internal and external questions}
Carnap already said a similar thing: the question ``what is there'' has external and internal readings.
Carnap (1956) ``Empiricism, Semantics and Ontology'' aims at a way of talking which is anti-metaphysical positivist friendly.
The question ``are they numbers?'' has two different answers: internal or tirival one, which deduced within a given language, and external one, which standard metaphysicains care and Carnap thinks meaningless and devoid.
Hofweber generally affirms Carnap.

\paragraph{3.5.4. Lightweight and Heavyweight Quantifiers}
Ordinary communication in daily life first. Metaphysicis later.

\section*{The Solution to the First Puzzle}
This section explains how the arguments  in this chapter give a solution to the first puzzle: how difficult to answer ontological questions?
\footnote{Recall his methodological notes in 1.5 [p. 18-19]. It is not enough to affirm either of these two. We need to specify the reason why such two different replies appear.}


 %----------------------------------------------------------------------------------------
%	BIBLIOGRAPHY
%----------------------------------------------------------------------------------------

\renewcommand{\refname}{Reference} % Change the default bibliography title

\bibliography{Mendeley} % Input your bibliography file

%----------------------------------------------------------------------------------------

\end{document}
